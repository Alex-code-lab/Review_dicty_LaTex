\usepackage{lipsum} % Pour générer du texte factice
\usepackage[utf8]{inputenc} % Encodage des caractères
\usepackage[T1]{fontenc} % Encodage de la police
\usepackage[french]{babel} % Langue principale du document
\usepackage{amsmath,amsfonts,amsthm} % Pour les mathématiques
\usepackage{graphicx} % Pour inclure des images
\usepackage{hyperref} % Pour les liens hypertextes
\usepackage{geometry} % Configuration des marges
\geometry{top=2cm, bottom=2cm, left=1.5cm, right=1.5cm, columnsep=20pt}
\usepackage{authblk} % Pour la gestion des auteurs et affiliations
\usepackage{titlesec} % Pour les titres de sections
\usepackage{abstract} % Personnalisation du résumé
\bibliographystyle{unsrt}  % Définition du style de bibliographie

% %% Allure générale du document
% \usepackage[notbib]{tocbibind}          % Intègre TOC, LOF, LOT dans la table des matières
% \pagestyle{headings}                    % Style de page avec en-têtes
% \usepackage{enumerate}                  % Pour les listes numérotées
% \usepackage{enumitem}                   % Plus de fonctionnalités pour les listes
% \usepackage[section]{placeins}          % Assure que les flottants restent dans leur section
% \usepackage{epigraph}                   % Pour ajouter des épigraphes
% \usepackage[font={small}]{caption}      % Options pour les légendes des figures/tables

% \usepackage{amsmath}                    % Pour l'écriture de mathématiques avancées
% \usepackage{amsthm}                     % Pour la définition de théorèmes et propositions
% \usepackage{bbold}                      % Pour les polices mathématiques en gras
% \usepackage[squaren,Gray]{SIunits}      % Pour l'écriture correcte des unités
% \usepackage{cite}                       % Pour la gestion des citations
% \usepackage{hyperref}                   % Pour la création de liens dans le document
% \usepackage{graphics}                   % Pour inclure des graphiques
% \usepackage{algorithmic}                % Pour écrire des algorithmes
% \usepackage{subfig}                     % Pour la gestion des sous-figures
% \usepackage{url}                        % Pour l'écriture correcte des URL
% \usepackage{xcolor}                     % Pour la gestion des couleurs

% Configuration du style des titres
\titleformat{\section}{\normalfont\large\bfseries}{\thesection}{1em}{}
\titleformat{\subsection}{\normalfont\normalsize\bfseries}{\thesubsection}{1em}{}

% Personnalisation de l'abstract
\renewenvironment{abstract}
 {\small
  {\bfseries \abstractname\vspace{-.5em}\vspace{10pt}\par}  % Titre en gras suivi d'un saut de paragraphe
  \noindent  % Assure que le texte commence sans indentation
 }
 {\par}
