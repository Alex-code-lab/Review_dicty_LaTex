%%%%%%%%%%%%%%%%%%%%%%%%%%%%%%%%%%%%%%%%%
%           Fichier maître              %
%%%%%%%%%%%%%%%%%%%%%%%%%%%%%%%%%%%%%%%%%

\documentclass[twocolumn,10pt]{article} % Utilisation du mode deux colonnes
\usepackage{lipsum} % Pour générer du texte factice
\usepackage[utf8]{inputenc} % Encodage des caractères
\usepackage[T1]{fontenc} % Encodage de la police
\usepackage[french]{babel} % Langue principale du document
\usepackage{amsmath,amsfonts,amsthm} % Pour les mathématiques
\usepackage{graphicx} % Pour inclure des images
\usepackage{hyperref} % Pour les liens hypertextes
\usepackage{geometry} % Configuration des marges
\geometry{top=2cm, bottom=2cm, left=1.5cm, right=1.5cm, columnsep=20pt}
\usepackage{authblk} % Pour la gestion des auteurs et affiliations
\usepackage{titlesec} % Pour les titres de sections
\usepackage{abstract} % Personnalisation du résumé
\bibliographystyle{unsrt}  % Définition du style de bibliographie

% %% Allure générale du document
% \usepackage[notbib]{tocbibind}          % Intègre TOC, LOF, LOT dans la table des matières
% \pagestyle{headings}                    % Style de page avec en-têtes
% \usepackage{enumerate}                  % Pour les listes numérotées
% \usepackage{enumitem}                   % Plus de fonctionnalités pour les listes
% \usepackage[section]{placeins}          % Assure que les flottants restent dans leur section
% \usepackage{epigraph}                   % Pour ajouter des épigraphes
% \usepackage[font={small}]{caption}      % Options pour les légendes des figures/tables

% \usepackage{amsmath}                    % Pour l'écriture de mathématiques avancées
% \usepackage{amsthm}                     % Pour la définition de théorèmes et propositions
% \usepackage{bbold}                      % Pour les polices mathématiques en gras
% \usepackage[squaren,Gray]{SIunits}      % Pour l'écriture correcte des unités
% \usepackage{cite}                       % Pour la gestion des citations
% \usepackage{hyperref}                   % Pour la création de liens dans le document
% \usepackage{graphics}                   % Pour inclure des graphiques
% \usepackage{algorithmic}                % Pour écrire des algorithmes
% \usepackage{subfig}                     % Pour la gestion des sous-figures
% \usepackage{url}                        % Pour l'écriture correcte des URL
% \usepackage{xcolor}                     % Pour la gestion des couleurs

% Configuration du style des titres
\titleformat{\section}{\normalfont\large\bfseries}{\thesection}{1em}{}
\titleformat{\subsection}{\normalfont\normalsize\bfseries}{\thesubsection}{1em}{}

% Personnalisation de l'abstract
\renewenvironment{abstract}
 {\small
  {\bfseries \abstractname\vspace{-.5em}\vspace{10pt}\par}  % Titre en gras suivi d'un saut de paragraphe
  \noindent  % Assure que le texte commence sans indentation
 }
 {\par}
		                   % Liste des packages et de leurs options

% Configuration des métadonnées du PDF
\hypersetup{
    pdfauthor={Alexandre Souchaud},
    pdfsubject={Review Dicty},
    pdftitle={Review Dicty},
    pdfkeywords={Biophysique, Review, migration, Dictyostelium}
}

% Configuration du titre de l'article
\title{Dicty's motility}

% Configuration des auteurs et leurs appartenances
\author[1]{A. Souchaud}
\author[2]{S. DeMonte}

\affil[1]{\small Institut de Biologie de l'ENS, France}
\affil[2]{\small Allemagne}

% Configuration de la date
\date{May 2024}  % Date de publication

% Début du document
\begin{document}

\maketitle  % Crée le titre du document

% \tableofcontents  % Crée la table des matières

\begin{abstract}
    Dictyostelium discoideum est un modèle particulièrement pertinent pour étudier la coopération et 
    le comportement collectif des cellules sous l'angle de l'évolution. La régulation génétique de
     l'agrégation cellulaire joue un rôle clé dans le comportement lié à la fitness des cellules.
      Cependant, les variations phénotypiques, influencées par l'environnement, ont également un impact
       significatif. Cette revue se concentrera sur l'impact des variations phénotypiques, en particulier 
       la motilité et l'adhésion cellulaire, sur la fitness. Elle examinera également comment l'altération de ces 
       caractéristiques par les conditions environnementales peut influencer l'évolution des traits au fil du temps. 
       En tenant compte de la variabilité des processus d'agrégation, nous explorerons les mécanismes par lesquels 
       ces traits phénotypiques influencent la fitness et comment ils peuvent être ciblés pour orienter l'évolution 
       de l'organisme.
\end{abstract}

\section{I. Weber et al. 1995}

During the growth phase and early development, cells of D. discouideum are extensively spread over a surface on which the move.
After 6+7 hours of starvation : cells become aggregation-competent (capacity of assembling into streams and responding chemotactically to cAMP)
This is accompained by distinc changes in cell shape and locomotion. Cells become elongated. This changes are accompagned 
by a dramatic reduction in size of the area of contact wtween cell and substratum. \cite{Weber_1995}

\textbf{Cell shape change upon contact with a substrate at the onset of the aggregation phase:}
Different studies of cell motility, where aggregation is possible, on surfaces that are moderately (BSA-coated glass surface) and highly adhesive (silanized glass), show that adhesion plays a role in the shape of the cells as well as in their biological activity (loss of parts of their membrane during movement), Schindl et al., 1995.

\textbf{Relationship between cell shape change and the contact surface with the substrate during the chemotaxis phase:}
Cyclic AMP influences the cells' response to adhesion: for instance, the competition between two pseudopods, one of which is not adherent and the other is. It is the one that is not adherent to the substrate that will eventually become the leading front of the cell.

\textbf{Motility of WT and mutant cells on different substrates:}
The AX2-WT cells do not seem to show differences in motility across different substrates (BSA coated and mica).
However, the mutant cells (lacking two F-actin crosslinking proteins) behave significantly differently on mica.

\section{T.J. Lampert et al.}
\cite{Weber_2017}
La plupart des choses que l'on connait sur la migration des amoeboid viennent de Dictyostelium discoideum.
C'est une connaissance et une étude essentiel car ce mode de motilité est celui retrouvé chez les cellules des métazoaires,
et également chez les cellules tumorales. 
Simple et partangeant de nombreux traits avec les métazoires, Dictyostelium est un outil formidable d'étude, physiologiquement
et génétiquement. 

\textbf{Focalisation sur la protéine PTEN:}
Les cellules dépourvue de la protéine tumorales PTEN montre une activité migratoire réduite, et une adhésion au substrat accrue. 

\textbf{Sélection basée sur l'adhérence :} 
Une méthode de criblage génétique utilisant l'intégration médiée par des enzymes de 
restriction a été appliquée pour générer des mutants de Dictyostelium, résultant en plusieurs souches présentant les phénotypes 
désirés en termes d'adhésion et de mobilité.

\textbf{Analyse des mutants :} 
Plusieurs souches ont montré une augmentation de l'adhérence et une réduction de la motilité. 
Des caractérisations supplémentaires de sept souches ont révélé une migration dirigée diminuée, une augmentation des 
protubérances basées sur l'actine filamentaire, et une activité accrue du réseau de signalisation.




\section{SCAR knockouts in Dictyostelium: Weltman 2012}
Les cellules eucaryotes migrent grâce à la formation de pseudopodes faits de filaments d'actine. \cite{Veltman_2012}


\section{A 30 year Perspective on microtubule-Based Motility in Dictyostelium }
\cite{Koonce_2020}
The review focuses on the MT-based set of motors and in the compact organism DD. 
Collective actions of the 13 kinesins and 1 dynein 
Motor isoforms = plusieurs configurations d’une protéine pour une spécialisation

MT-based motilities : Most of the visual motility is dependent on the MT cytoskeleton and carefully quantitated in
 \cite{Roos_1987}. Les organelles sont mues par les MT, mais également durant les interphases, les MT sont essentielles. 

 \subsubsection*{Dynein}
its deletion is lethal, and genome analysis demonstrate a single isoform of the minus-MT-end-directed cytoplasmic dynein in DD. 
\subsubsection*{Kinesin}
coucou
\subsubsection*{Developmentally regulated}
D.D. has a vegetative growth stage, where a single amoeba crawls, feeds and divides. Starvation triggers a cAMP signaling cascade to aggregate cells into groups of about 10\^{}5 cells and initiates a developmental program to form spore-filled capsules lifted off the substrate on the top of stalks.
spore-filled capsules lifted off the substrate on the top of stalks.
DdKif2, ddKif7 (Kinesin 14 and Kinesin-1) do not appear to be expressed durung vegetative groxwth but mRNAs are present after 8h of starvation. 
Gene knockouts of either motor do no reveal any significant vegetative cell defects.\cite{deHostos_1998}.

\subsubsection*{Discussion}
Au vu des conclusion on peut voir que Dynein jouant un rôle prépondérant dans les interphase et activités de la meiose,
 il est un bon candidat de marquage pour ces activités (sous réserve de faire un mutant fluo et que la production de 
 Dynein soit suffisiament plus éleveées à ces moments).

\bibliography{../Bibliographie/bibliographie}  % Inclut la bibliographie

\end{document}
