%%%%%%%%%%%%%%%%%%%%%%%%%%%%%%%%%%%%%%%%%
%           Fichier maître              %
%%%%%%%%%%%%%%%%%%%%%%%%%%%%%%%%%%%%%%%%%

\documentclass[twocolumn,10pt]{article} % Utilisation du mode deux colonnes
\usepackage{lipsum} % Pour générer du texte factice
\usepackage[utf8]{inputenc} % Encodage des caractères
\usepackage[T1]{fontenc} % Encodage de la police
\usepackage[french]{babel} % Langue principale du document
\usepackage{amsmath,amsfonts,amsthm} % Pour les mathématiques
\usepackage{graphicx} % Pour inclure des images
\usepackage{hyperref} % Pour les liens hypertextes
\usepackage{geometry} % Configuration des marges
\geometry{top=2cm, bottom=2cm, left=1.5cm, right=1.5cm, columnsep=20pt}
\usepackage{authblk} % Pour la gestion des auteurs et affiliations
\usepackage{titlesec} % Pour les titres de sections
\usepackage{abstract} % Personnalisation du résumé
\bibliographystyle{unsrt}  % Définition du style de bibliographie

% %% Allure générale du document
% \usepackage[notbib]{tocbibind}          % Intègre TOC, LOF, LOT dans la table des matières
% \pagestyle{headings}                    % Style de page avec en-têtes
% \usepackage{enumerate}                  % Pour les listes numérotées
% \usepackage{enumitem}                   % Plus de fonctionnalités pour les listes
% \usepackage[section]{placeins}          % Assure que les flottants restent dans leur section
% \usepackage{epigraph}                   % Pour ajouter des épigraphes
% \usepackage[font={small}]{caption}      % Options pour les légendes des figures/tables

% \usepackage{amsmath}                    % Pour l'écriture de mathématiques avancées
% \usepackage{amsthm}                     % Pour la définition de théorèmes et propositions
% \usepackage{bbold}                      % Pour les polices mathématiques en gras
% \usepackage[squaren,Gray]{SIunits}      % Pour l'écriture correcte des unités
% \usepackage{cite}                       % Pour la gestion des citations
% \usepackage{hyperref}                   % Pour la création de liens dans le document
% \usepackage{graphics}                   % Pour inclure des graphiques
% \usepackage{algorithmic}                % Pour écrire des algorithmes
% \usepackage{subfig}                     % Pour la gestion des sous-figures
% \usepackage{url}                        % Pour l'écriture correcte des URL
% \usepackage{xcolor}                     % Pour la gestion des couleurs

% Configuration du style des titres
\titleformat{\section}{\normalfont\large\bfseries}{\thesection}{1em}{}
\titleformat{\subsection}{\normalfont\normalsize\bfseries}{\thesubsection}{1em}{}

% Personnalisation de l'abstract
\renewenvironment{abstract}
 {\small
  {\bfseries \abstractname\vspace{-.5em}\vspace{10pt}\par}  % Titre en gras suivi d'un saut de paragraphe
  \noindent  % Assure que le texte commence sans indentation
 }
 {\par}
		                   % Liste des packages et de leurs options

% Configuration des métadonnées du PDF
\hypersetup{
    pdfauthor={Alexandre Souchaud},
    pdfsubject={Review Dicty},
    pdftitle={Review Dicty},
    pdfkeywords={Biophysique, Review, migration, Dictyostelium}
}

% Configuration du titre de l'article
\title{Dicty's motility}

% Configuration des auteurs et leurs appartenances
\author[1]{A. Souchaud}
\author[2]{S. DeMonte}

\affil[1]{\small Institut de Biologie de l'ENS, France}
\affil[2]{\small Allemagne}

% Configuration de la date
\date{May 2024}  % Date de publication

% Début du document
\begin{document}

\maketitle  % Crée le titre du document

% \tableofcontents  % Crée la table des matières

\begin{abstract}
    Dictyostelium discoideum est un modèle particulièrement pertinent pour étudier la coopération et 
    le comportement collectif des cellules sous l'angle de l'évolution. La régulation génétique de
     l'agrégation cellulaire joue un rôle clé dans le comportement lié à la fitness des cellules.
      Cependant, les variations phénotypiques, influencées par l'environnement, ont également un impact
       significatif. Cette revue se concentrera sur l'impact des variations phénotypiques, en particulier 
       la motilité et l'adhésion cellulaire, sur la fitness. Elle examinera également comment l'altération de ces 
       caractéristiques par les conditions environnementales peut influencer l'évolution des traits au fil du temps. 
       En tenant compte de la variabilité des processus d'agrégation, nous explorerons les mécanismes par lesquels 
       ces traits phénotypiques influencent la fitness et comment ils peuvent être ciblés pour orienter l'évolution 
       de l'organisme.
\end{abstract}

\section{Introduction}
La compréhension des processus évolutifs et développementaux est un défi central en biologie,
nécessitant une exploration des mécanismes à la fois immédiats et historiques qui façonnent les organismes.
Deux concepts fondamentaux émergent dans cette perspective : les causes proximales et les causes ultimes.
Les causes proximales se réfèrent aux mécanismes biologiques immédiats responsables des traits observés,
tandis que les causes ultimes traitent des raisons évolutives et historiques pour lesquelles ces traits existent.
Ces deux types de causes ont été au cœur de nombreux débats scientifiques, depuis les premiers travaux d’Ernst Mayr (1961)
qui les a introduits comme une distinction clé pour comprendre les processus biologiques. Mayr a souligné que les
causes proximales concernent les mécanismes physiologiques immédiats influençant le développement et le comportement 
d'un organisme, tandis que les causes ultimes examinent les raisons évolutives qui expliquent pourquoi un certain trait 
ou comportement a été favorisé par la sélection naturelle.
Cependant, relier ces deux types de causes reste une tâche complexe, nécessitant des modèles biologiques qui permettent 
d'étudier leur interaction de manière intégrée. David Haig (2013) a approfondi cette distinction en explorant comment les 
causes proximales et ultimes peuvent être reliées pour comprendre non seulement comment les traits biologiques se manifestent, 
mais aussi pourquoi ils persistent au fil de l'évolution. C’est dans ce contexte que Dictyostelium discoideum, 
un amibozoaire unicellulaire, se révèle être un modèle exceptionnellement utile.

Dictyostelium discoideum est souvent qualifié d'organisme modèle en raison de sa simplicité relative, 
de son cycle de vie unique, et de sa capacité à former des structures multicellulaires en réponse à des signaux environnementaux. 
Ce cycle de vie comporte deux phases principales : une phase unicellulaire où les amibes se nourrissent de bactéries dans le sol, 
et une phase multicellulaire induite par le stress, comme le manque de nourriture, où des milliers de cellules individuelles 
s'agrègent pour former une structure multicellulaire appelée pseudoplasmode (DeMonte, 2022). Cette transition entre des modes 
de vie unicellulaires et multicellulaires permet d’explorer des questions fondamentales sur la coopération, la différenciation 
cellulaire, et les conflits sociaux, tout en offrant des perspectives uniques pour étudier les interactions entre causes proximales 
et ultimes (Ostrowski, 2019).

\textbf{Les Avantages de Dictyostelium comme Modèle d'Étude}
L'utilisation de Dictyostelium permet une étude approfondie des causes proximales et ultimes dans un système simple mais représentatif des processus évolutifs complexes observés dans des organismes multicellulaires plus évolués. Par exemple, le comportement de coopération observé lors de la formation du pseudoplasmode — où certaines cellules se sacrifient pour former une tige qui soutient un corps fructifère contenant des spores — peut être analysé à travers le prisme des causes ultimes, en termes d'évolution de la coopération et de la sélection de parenté (Medina, 2019). Les mécanismes génétiques et moléculaires sous-jacents à ces comportements, qui représentent les causes proximales, peuvent être explorés grâce aux outils de biologie moléculaire et de génétique disponibles pour ce modèle (Ostrowski et al., 2015).
De plus, Dictyostelium permet l'étude de conflits sociaux, notamment entre cellules coopératives et tricheuses. Les cellules tricheuses exploitent souvent la coopération des autres sans contribuer elles-mêmes, un comportement qui pose des questions fascinantes sur la stabilité de la coopération dans des populations mixtes. Ces questions ne sont pas seulement pertinentes dans le contexte de Dictyostelium, mais ont également des implications pour la compréhension des conflits sociaux et de la coopération dans un large éventail d’organismes, y compris les humains. Ainsi, Dictyostelium devient un modèle de choix pour explorer des questions biologiques universelles (Kümmerli et al., 2015; Gilbert et al., 2007).

\textbf{L'Interface entre Causes Proximales et Ultimes}
L'une des forces de Dictyostelium en tant que modèle est sa capacité à être étudié sur plusieurs échelles de temps. 
Les causes proximales, telles que les réponses immédiates aux signaux de faim ou les mécanismes moléculaires de l’adhésion 
cellulaire, peuvent être observées et manipulées dans des conditions contrôlées en laboratoire (Sawada et Chubb). En parallèle,
 les causes ultimes, y compris les pressions de sélection qui ont façonné les stratégies de vie de Dictyostelium sur des milliers
  de générations, peuvent être explorées par des études évolutionnaires et comparatives (Cooper et West, 2022).
Les recherches récentes ont commencé à éclairer comment ces deux types de causes interagissent et s'influencent mutuellement.
 Par exemple, les dynamiques éco-évolutives de Dictyostelium montrent comment les interactions sociales au niveau cellulaire
  peuvent influencer les trajectoires évolutives à long terme. Des comportements,tels que la tricherie, peuvent être favorisés
   ou contrés par la structure de la population et les conditions environnementales, illustrant une interaction complexe entre
    causes proximales et ultimes (Patten et al., 2023). Ces études suggèrent que pour comprendre pleinement les processus évolutifs
     et développementaux, il est crucial de considérer ces niveaux de causalité de manière intégrée.

\textbf{Motilité et Adhésion comme Outils pour Comprendre les Enjeux Évolutifs}
Un aspect particulièrement pertinent de l'étude de Dictyostelium concerne sa motilité et son adhésion cellulaire. 
La capacité de mouvement et d'adhésion des cellules joue un rôle crucial dans la phase d'agrégation, influençant directement 
la fitness des cellules individuelles et, par extension, la survie de la population entière. La motilité de Dictyostelium est 
un excellent modèle pour comprendre comment des facteurs proximaux, tels que les signaux chimiotactiques et les changements dans 
l'organisation du cytosquelette, peuvent conduire à des comportements coordonnés complexes qui ont des implications évolutives
 profondes. Les recherches de Weber et al. (1995) ont montré que les changements de forme cellulaire associés à la motilité sont
  étroitement liés à l'adhésion au substrat, ce qui affecte la capacité des cellules à se déplacer et à s'agréger efficacement. 
  De même, des études sur des mutants de Dictyostelium (Veltman, 2012) ont révélé que les altérations dans les protéines clés de 
  la motilité, comme celles impliquées dans le cytosquelette d'actine, peuvent modifier la capacité de migration et d'adhésion des 
  cellules, affectant ainsi leur survie et leur succès reproducteur.
La revue explorera comment la motilité et l'adhésion, en tant que traits phénotypiques influencés par des conditions 
environnementales, peuvent servir de pont pour relier les causes proximales et ultimes. Comprendre ces mécanismes peut fournir 
des informations essentielles sur la manière dont les organismes résolvent des problèmes évolutifs complexes tels que la coopération 
et la tricherie.


\textbf{Objectif de cette Revue}
Cette revue vise à explorer comment Dictyostelium discoideum peut servir de modèle pour l'étude intégrée des causes proximales 
et ultimes en biologie. En utilisant Dictyostelium comme cadre de référence, nous examinerons les travaux récents sur les 
comportements sociaux, la coopération, et les conflits, et comment ces processus sont influencés à la fois par les mécanismes
 génétiques immédiats et par les pressions évolutives à long terme. Nous discuterons également des défis et des opportunités 
 de relier ces deux niveaux de causalité, offrant ainsi des perspectives sur la manière dont les études sur Dictyostelium
  peuvent éclairer des questions plus larges en biologie évolutive et développementale.


\section{I. Weber et al. 1995}

During the growth phase and early development, cells of D. discouideum are extensively spread over a surface on which the move.
After 6+7 hours of starvation : cells become aggregation-competent (capacity of assembling into streams and responding chemotactically to cAMP)
This is accompained by distinc changes in cell shape and locomotion. Cells become elongated. This changes are accompagned 
by a dramatic reduction in size of the area of contact wtween cell and substratum. \cite{Weber_1995}

\textbf{Cell shape change upon contact with a substrate at the onset of the aggregation phase:}
Different studies of cell motility, where aggregation is possible, on surfaces that are moderately (BSA-coated glass surface) and highly adhesive (silanized glass), show that adhesion plays a role in the shape of the cells as well as in their biological activity (loss of parts of their membrane during movement), Schindl et al., 1995.

\textbf{Relationship between cell shape change and the contact surface with the substrate during the chemotaxis phase:}
Cyclic AMP influences the cells' response to adhesion: for instance, the competition between two pseudopods, one of which is not adherent and the other is. It is the one that is not adherent to the substrate that will eventually become the leading front of the cell.

\textbf{Motility of WT and mutant cells on different substrates:}
The AX2-WT cells do not seem to show differences in motility across different substrates (BSA coated and mica).
However, the mutant cells (lacking two F-actin crosslinking proteins) behave significantly differently on mica.

\section{T.J. Lampert et al.}
\cite{Weber_2017}
La plupart des choses que l'on connait sur la migration des amoeboid viennent de Dictyostelium discoideum.
C'est une connaissance et une étude essentiel car ce mode de motilité est celui retrouvé chez les cellules des métazoaires,
et également chez les cellules tumorales. 
Simple et partangeant de nombreux traits avec les métazoires, Dictyostelium est un outil formidable d'étude, physiologiquement
et génétiquement. 

\textbf{Focalisation sur la protéine PTEN:}
Les cellules dépourvue de la protéine tumorales PTEN montre une activité migratoire réduite, et une adhésion au substrat accrue. 

\textbf{Sélection basée sur l'adhérence :} 
Une méthode de criblage génétique utilisant l'intégration médiée par des enzymes de 
restriction a été appliquée pour générer des mutants de Dictyostelium, résultant en plusieurs souches présentant les phénotypes 
désirés en termes d'adhésion et de mobilité.

\textbf{Analyse des mutants :} 
Plusieurs souches ont montré une augmentation de l'adhérence et une réduction de la motilité. 
Des caractérisations supplémentaires de sept souches ont révélé une migration dirigée diminuée, une augmentation des 
protubérances basées sur l'actine filamentaire, et une activité accrue du réseau de signalisation.




\section{SCAR knockouts in Dictyostelium: Weltman 2012}
Les cellules eucaryotes migrent grâce à la formation de pseudopodes faits de filaments d'actine. \cite{Veltman_2012}


\section{A 30 year Perspective on microtubule-Based Motility in Dictyostelium }
    \cite{Koonce_2020}
    The review focuses on the MT-based set of motors and in the compact organism DD. 
    Collective actions of the 13 kinesins and 1 dynein 
    Motor isoforms = plusieurs configurations d’une protéine pour une spécialisation

    MT-based motilities : Most of the visual motility is dependent on the MT cytoskeleton and carefully quantitated in
    \cite{Roos_1987}. Les organelles sont mues par les MT, mais également durant les interphases, les MT sont essentielles. 

    \subsubsection*{Dynein}
    its deletion is lethal, and genome analysis demonstrate a single isoform of the minus-MT-end-directed cytoplasmic dynein in DD. 
    \subsubsection*{Kinesin}
    coucou
    \subsubsection*{Developmentally regulated}
    D.D. has a vegetative growth stage, where a single amoeba crawls, feeds and divides. Starvation triggers a cAMP signaling cascade to aggregate cells into groups of about 10\^{}5 cells and initiates a developmental program to form spore-filled capsules lifted off the substrate on the top of stalks.
    spore-filled capsules lifted off the substrate on the top of stalks.
    DdKif2, ddKif7 (Kinesin 14 and Kinesin-1) do not appear to be expressed durung vegetative groxwth but mRNAs are present after 8h of starvation. 
    Gene knockouts of either motor do no reveal any significant vegetative cell defects.\cite{deHostos_1998}.

    \subsubsection*{Discussion}
    Au vu des conclusion on peut voir que Dynein jouant un rôle prépondérant dans les interphase et activités de la meiose,
    il est un bon candidat de marquage pour ces activités (sous réserve de faire un mutant fluo et que la production de 
    Dynein soit suffisiament plus éleveées à ces moments).

\section{Introduction}
Objectif est de montrer que Dictyostelium peut servir de lien entre les causes proximales et les causes ultimes :
le traitement de la littérature traite surtout les causes ultimes avec la biologie évolutive classique.
   
   
    \subsection{Définition des concepts}
    \cite{Haig_2003}
    \cite{Mayr_1961}
    
            \subsubsection{Causes immédiates / proximales}
               Ces causes se réfèrent aux mécanismes immédiats et directs qui influencent le développement et le comportement des organismes. Par exemple, les réponses physiologiques, les réactions biochimiques et les processus neuronaux.
               En biologie fonctionnelle, ces causes expliquent comment un trait ou un comportement se manifeste.
               Exemple : La libération de l'hormone adrénaline en réponse au stress est une cause immédiate de l'accélération du rythme cardiaque.
   
               \subsubsection{causes ultimes}
               Elles se rapportent aux explications historiques et évolutionnaires qui ont conduit à l'existence de ces mécanismes. Elles s'intéressent aux raisons pour lesquelles ces traits ou comportements ont été favorisés par la sélection naturelle.
               En biologie évolutionnaire, ces causes expliquent pourquoi un trait ou un comportement a évolué en termes de survie et de reproduction.
               Exemple : L'augmentation du rythme cardiaque en réponse au stress peut être expliquée par la nécessité évolutive de préparer le corps à une réaction de lutte ou de fuite, améliorant ainsi les chances de survie.
   
               
               \subsubsection{distinction faite par Haig par rapport à Mayr}
               A noter que Haig recommande une utilisation plus détaillée des causes ultimes qui portent à confusion (volontairement ou non) sur les "objectifs de l'évolution". 
               Plutôt que cause ultime, il peut etre utilisé causes efficientes (concentration sur les mécanismes) et causes finales (pour les fonctions adaptatives).
               En effet, la sépération du comment (les mécanismes donc les causes efficientes / proximales) avec le "pourquoi" (fonction adatptatives, donc finales) 
               se fait car elles apportent des réponses différentes et complétementaires, selon lui. 
   
               \subsubsection{Pour rappel...}
               Ernst Mayr (1904-2005) biologiste évolutif allemand (théorie synthétique de l'évolution mélant Darwin et génétique) et a, entre autre, défini biologiquement l'espèce
   
    \subsection{definition de la triche et de la coopération}
                \subsubsection{Points de rappel sur la définition de la triche qui, à mon sens, est mal employée}
                    \begin{itemize}
                        \item \textbf{Définition Générale :} Dans la biologie sociale, la triche fait référence à des comportements où certains individus
                        bénéficient disproportionnellement des efforts coopératifs des autres, sans contribuer équitablement en retour.
                        \item \textbf{Contexte dans Dictyostelium :} Ici, cela signifie que certaines cellules augmentent leurs chances de devenir des
                        spores, maximisant ainsi leur propre succès reproductif, souvent en réduisant la probabilité que d'autres
                        cellules deviennent des spores.
                    \end{itemize}

                    finalement, le terme de triche (et donc tricheur) permet de caractériser les comportements (ou les individus) maximisant
                    leur succès reproducteur au détriment apparent des autres. Le terme triche peut porter à confusion car il sous-entend un caractère
                    volontaire et prémédité. Dans le cas de Dictyostelium il s'agit d'une adaptation naturelle aux pressions de séléction. 
                    "Comportement opportuniste" me semble plus adapté pour décrire la situation et perdre l'ambiguïté. 

                \subsubsection{coopération:}
                    \begin{itemize}
                        \item \textbf{Définition Générale :} Comportements où les individus travaillent ensemble pour un bénéfice mutuel ou collectif.
                        \item \textbf{Contexte dans Dictyostelium :} Les cellules coopèrent pour former des structures multicellulaires où certaines 
                        deviennent des cellules tiges (sacrifiées) et d'autres des spores (cellules reproductrices).
                    \end{itemize}

                \subsubsection{note:}
                    \begin{itemize}
                        \item \textbf{Importance du goulot d'étranglement génétique} (ou unicellulaire) : événement durant lequel il y'a une diminution drastique de la population (facteurs
                        environnementaux, maladies, ...). Cette diminution a d'importantes conséquences sur la suite évolutive de la population : le matériel génétique disponible
                        est largement diminué, et donc sera la base de la nouvelle population.
                        Ce goulot d'étranglement diminue donc la diversité génétique et peut être propice à l'émergence de la multicellularité. En effet, avec une diminution des conflits
                        génétiques au sein de la population, la coopération est favorisée car les cellules partagent un intérêt évolutif commun, et donc les agrégats multicellulaires sont plus suceptible d'apparaitre.
                        (c'est vrai ça? Par exemple, dans la symbiose...).
                        Egaglement, cela peut permettre une purge des mutations délétères. 
                        Enfin, on a l'émérgence de nouveaux traits bénéfiques parmis les cellules survivantes qui sont favorisées et qui émergent alors.

                        Au final, l'agrégation lors d'une famine de DD, peut être vu comme un goulot d'étranglement. 


                        \item \textbf{Comprendre les mécanismes d'agrégation chez Dicty} c'est comprendre des mécanismes cellulaires fondamentaux puisque 
                        l'apparation de la multicellularité s'est faites plusieurs fois, de façon indépendantes, et dans des clades différentes.
                        C'est donc qu'il existe des mécanismes sous-jacents qui peuvent se reproduire. Les comprendres, c'est comprendre des mécanismes
                        fondamentaux cellulaires.

                        \item \textbf{De plus, DD est un système modèle expérimental} facilement accessible et dont les temps de reproduction offre une large possibilité
                        expérimentale. Notamment, il est possible d'étudier les causes proximales comme les causes ultimes. En observant les cycles de vie
                        de DD, ont peut étudier la façon dont la selection naturelle agit sur les traits coopératifs, altruistes, et tricheurs. Ces connaissances
                        dépassent le cadre de la biologie pure, puisque en plus d'être applicables à d'autres systèmes, ils sont également utilisables pour des modèles
                        de comportements sociaux humains.

                        \item \textbf{L'étude des interactions entre cellules coopératives} et tricheuses chez Dictyostelium aide à comprendre comment 
                        les organismes multicellulaires résolvent les conflits génétiques. Cela est pertinent pour des domaines allant de 
                        l'évolution des systèmes immunitaires à la biologie des cancers, où des cellules tricheuses (cancéreuses) 
                        exploitent les ressources de l'organisme.


                        \item \textbf{Implications pour la Biologie du Développement :} Les processus de différenciation cellulaire chez Dictyostelium, où 
                        certaines cellules deviennent des spores et d'autres des cellules de tige, offrent des parallèles intéressants avec 
                        les processus de développement des organismes multicellulaires plus complexes.
                        Cela peut aider à comprendre les mécanismes de régulation génétique et épigénétique impliqués dans la différenciation cellulaire.
                    \end{itemize}

                    \subsubsection{note bis: pourquoi les modèles théoriques}
                    \begin{itemize}
                        \item \textbf{Simplification et abstractions :} concentrations sur les intéractions essencielles et donc définitions des variables essentielles.
                        \item \textbf{Prediction testables :} faires des prédictions pour les tester expérimentalement et conclure.
                        \item \textbf{Intégrations de données de différentes sources : } Les données de différents labo, expériences etc peuvent être mis en commun pour 
                        affiner les résultats
                        
                    \end{itemize}


\bibliography{../Bibliographie/bibliographie}  % Inclut la bibliographie

\end{document}
